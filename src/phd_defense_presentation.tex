\documentclass{beamer}
\usetheme{metropolis} % Use metropolis theme

\usepackage[a-2u]{pdfx}
\usepackage[czech]{babel}
\usepackage[utf8]{inputenc}
\usepackage[T1]{fontenc}
\usepackage{lmodern}
\usepackage{textcomp}
\usepackage{sansmath}  % enables switching to serif fonts also in math env
\usepackage{chemformula}
\usepackage{chemfig}
\usepackage[
	font=tiny,
	labelfont={tiny}
]{caption}
\usepackage{tabularx}  % tables with fixed width
\usepackage{booktabs}  % better looking tables, more space after horizontal
% rule (use \midrule) and possibility of bold line usage (\toprule and
% \bottomrule)

%%% tikz
\usepackage{tikz}  % nice images in LaTeX
\usetikzlibrary{%
	arrows.meta,%
	shapes,%
	intersections,%
	decorations.markings,%  arrows
	decorations.pathmorphing,%  zigzag, snake and other paths
	calc,%
	positioning%  left = 1cm and 3cm of SomeNode
}

% TikZ macros
\newcommand{\mypgfextractangle}[3]{%
	\pgfmathanglebetweenpoints{%
		\pgfpointanchor{#2}{center}}{%
		\pgfpointanchor{#3}{center}}
	\global\let#1\pgfmathresult
}

\newcommand{\icm}{cm$^{-1}$}  % reciprocal centimeters
%%% upgright greek letters
\newcommand{\g}[1]{\foreignlanguage{greek}{#1}}


\title{Studium struktury a interakcí nukleových kyselin pomocí rezonančního
Ramanova rozptylu}
\date{}
%\date{15. prosince 2021}
%\author{Mgr. Jakub Klener}
%\institute{%
	%\parbox[c]{0.14\textwidth}{%
		%\includegraphics[width=0.12\textwidth]{assets/logo.pdf}%
	%}%
	%\parbox[c]{0.8\textwidth}{Matematicko-fyzikální\\fakulta\\Univerzita Karlova}
%}
\author{%
	\parbox[c]{0.25\textwidth}{%
		\includegraphics[width=0.22\textwidth]{assets/logo.pdf}%
	}%
	\parbox[c]{0.7\textwidth}{%
		Jakub Klener\\%
		Matematicko-fyzikální fakulta\\%
		Univerzita Karlova\\%
		15. prosince 2021
	}%
}

\begin{document}

\maketitle

\begin{frame}{Obsah}
  \setbeamertemplate{section in toc}[sections numbered]
  \tableofcontents[hideallsubsections]
\end{frame}


\section{Rezonanční Ramanův rozptyl a nukleové kyseliny}

\begin{frame}{Nukleové kyseliny}
\begin{itemize}
	\item Je tvořena kyselinou fosforečnou, pentózou a nukleovými bázemi
	\item Zabezpečují uchování a účastní se exprese genetické informace
	\item Mohou zaujmout mnoho různých konformací
\end{itemize}

\begin{figure}
	\centering
	\includegraphics[width=.6\columnwidth]{assets/dna_conformations}
	\caption*{
		Pandia N. et al., 2021.
		\emph{Biochim. Biophis. Acta.}
		\textbf{1876}(2), 188594.
	}
\end{figure}
\end{frame}

\begin{frame}{Ramanova spektroskopie}
\begin{center}
	\resizebox{1\linewidth}{!}{\begin{tikzpicture}[line width=2, scale=1]
	\draw (0,0) -- ++(10.2,0);
	\draw (0,.6) -- ++(10.2,0);
	\draw (0,1.2) -- ++(10.2,0);
	\foreach \x in {0,.4, ...,10}
	{
		\draw [gray] (\x,4.5) -- ++(.2,0);
	}
	\draw (0,6) -- ++(10.2,0);
	\draw (0,6.6) -- ++(10.2,0);
	\draw (0,7.2) -- ++(10.2,0);

	\draw [->] (.5,0) -- ++(0,4.5);
	\draw [->] (1.5,4.5) -- ++(0,-4.5);

	\draw [->] (3.5,0) -- ++(0,4.5);
	\draw [->, color=red] (4.5,4.5) -- ++(0,-3.9);

	\draw [->] (6.5,0) -- ++(0,6.6);
	\draw [->, color=red] (7.5,6.6) -- ++(0,-6);

	\draw(1,0) node [below] {Rayleigh};
	\draw(4,0) node [below] {\textcolor{red}{Raman}};
	\draw(7,0) node [below] {\textcolor{red}{Rezonanční Raman}};
	\draw(0,4.5) node [left, align=right] {Virtuální\\hladiny};
	\draw(0,0.6) node [left, align=right] {Základní\\elektronový\\stav};
	\draw(0,6.6) node [left, align=right] {Excitovaný\\elektronový\\stav};

\end{tikzpicture}
}
\end{center}
\end{frame}

\begin{frame}{Proč UV rezonanční Ramanova spektroskopie?}
\begin{itemize}
	\item Zesílení, typicky o 1 -- 2 řády
	\item Zjednodušení spektra díky selektivnímu zesílení
	\item Hlavní třídy biomolekul mají elektronový absorpční přechod v~UV
	\item Fluorescence hlavně nečistot je i díky rezonančnímu přenosu často na
		větších vlnových délkách
\end{itemize}
\end{frame}


\section{Stavba spektrometru}

\begin{frame}{Design spektrometru}
\begin{center}
	\resizebox{1\linewidth}{!}{\input{backscattering_schema}}
\end{center}
\end{frame}

\begin{frame}{Kalibrace}
\begin{center}
	\resizebox{1\linewidth}{!}{\input{pt_calibration/pt_calibration}}
\end{center}
\end{frame}

\begin{frame}{Detail rotační cely}
\begin{center}
	\resizebox{.5\linewidth}{!}{\input{spinning_cell_drawing}}
\end{center}
\end{frame}

\begin{frame}{Detail termostatovaného držáku}
\begin{center}
	\includegraphics[width=.49\columnwidth]%
			{thermostated_holder_drawing/3D01}
	\includegraphics[width=.49\columnwidth]%
		{thermostated_holder_drawing/3D02}
\end{center}
\end{frame}

\begin{frame}{Optimalizace měření -- odečet intenzity}
\begin{center}
	\resizebox{1\linewidth}{!}{\input{power_optimization_triplexes/%
		power_optimization_triplexes_pU}}
\end{center}
\end{frame}

\begin{frame}{Optimalizace měření -- pokles v čase}
\begin{center}
	\resizebox{1\linewidth}{!}{\input{power_optimization_triplexes/%
		power_optimization_triplexes}}
\end{center}
\end{frame}

\begin{frame}{Interpretace spekter}
\begin{center}
	\resizebox{1\columnwidth}{!}{\input{interpretation/at/interpretation_at}}
\end{center}
\end{frame}

\begin{frame}{Interpretace spekter}
\begin{center}
	\includegraphics[width=1\columnwidth,page=1,trim={2cm 13.6cm 0 4cm},clip]%
		{interpretation/at/interpretation_table_at}
\end{center}
\end{frame}


\section{Použití spektrometru ke studiu nukleových kyselin}

\begin{frame}{Komplexy PolyA a PolyU}
\begin{center}
	\resizebox{1\columnwidth}{!}{\input{rna_triplex/rna_triplex_spectra}}
\end{center}
\end{frame}

\begin{frame}{Komplexy PolyA a PolyU}
\begin{center}
	$2 (\text{polyA} \cdot \text{polyU}) \rightarrow
		\text{polyU} \cdot \text{polyA} \cdot \text{polyU} + \text{polyA}$
\end{center}
\begin{center}
	\resizebox{1\columnwidth}{!}{\input{rna_triplex/rna_triplex_mg}}
\end{center}
\end{frame}

\begin{frame}{Komplexy PolyA a PolyU}
\begin{center}
	\resizebox{1\columnwidth}{!}{\input{rna_triplex/rna_triplex_conc}}
\end{center}
\end{frame}

\begin{frame}{Analýza vlivu teploty na strukturu DNA}
\begin{center}
	\includegraphics[width=1\columnwidth]{dna_hairpins/structure}
	\includegraphics[width=.9\columnwidth]{dna_hairpins/melting_spectra}
\end{center}
\end{frame}

\begin{frame}{Analýza vlivu teploty na strukturu DNA}
\begin{center}
	\includegraphics[width=.9\columnwidth]{dna_hairpins/pca_hairpin}
\end{center}
\end{frame}

\begin{frame}{Analýza vlivu teploty na strukturu DNA}
\begin{center}
	\includegraphics[width=.9\columnwidth]{dna_hairpins/forms_spectra}
\end{center}
\end{frame}

\begin{frame}{Analýza vlivu teploty na strukturu DNA}
\begin{figure}
	\centering
	\includegraphics[width=1\columnwidth]{tel22/quadruplex_forms}\\
	\caption*{
		Benabou S. et al., 2019.
		\emph{Sci. Rep.}
		\textbf{9}(15807), 1.
	}
\end{figure}
\end{frame}

\begin{frame}{Studium přechodu mezi paralelním a antiparalelním\\ kvadruplexem}
\begin{center}
	\resizebox{1\columnwidth}{!}{\input{tel22/tel22_spectra}}
\end{center}
\end{frame}


\refstepcounter{section}% Increase section counter
\sectionmark{Shrnutí}% Add section mark (header)
\addcontentsline{toc}{section}{\protect\numberline{}\thesection. Shrnutí\par}%

\begin{frame}{Shrnutí}
\begin{itemize}
	\item Byl postaven UV Ramanův spektrometr
	\begin{itemize}
		\item Byla vyvinuta metoda kalibrace vlnočtové škály
		\item Možnost měření v pravoúhlé geometrii i geometrii zpětného rozptylu
		\item Zkonstruována rotační cela
		\item Zkonstruován držák na kyvety umožňující teplotně závislá měření
		\item Optimalizovány podmínky měření
	\end{itemize}
	\item Vytvořena interpretační tabulka rezonančních Ramanových spekter
		nukleových kyselin
	\item Nalezená metodologie byla aplikována na několik strukturních studií
		nukleových kyselin
\end{itemize}
\end{frame}


\end{document}